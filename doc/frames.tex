\section{Frames}
\subsection{Definition}
Simpy put, a frame is a coordinate system. It defines the orientation and rotation of it's axis, as well as the position and
velocity of the origin relative to its' parent. A frame can be represented as the four-tuple
$\langle \left[\mathbf{Q}_{p\to l}\right], \mathbf{\omega}, \mathbf{r}_f, \mathbf{v}_f \rangle$.
$\left[\mathbf{Q}_{p\to l}\right]$ is the orthoginal translation from parent space to local space.
$\mathbf{\omega}$ is the angular velocity around the frames origin. $\mathbf{r}_f$ is the position of the origin, and $\mathbf{v}_f$
is its' velocity.

\subsubsection{Transformation to Parent Frame}
Given the state vector $\langle \mathbf{r}_l, \mathbf{v}_l \rangle$ in the local frame $F$, and the time since epoch $\Delta$,
elements of the state vector $\langle \mathbf{r}_p, \mathbf{v}_p \rangle$ in the parent frame $P$ are
\begin{equation}
    \label{frame_local_to_parent_r}
    \mathbf{r}_p=\mathbf{r}_{F}+\left[
            \mathbf{Q}_{F}^{-1}\mathbf{R}_{\hat{\mathbf{\omega}}_{F}}\left(|\omega_{F}|\Delta{}t\right)
        \right]\mathbf{r}_l
\end{equation}
\begin{equation}
    \label{frame_local_to_parent_v}
    \mathbf{v}_p=\mathbf{v}_{F}+\left[
            \mathbf{Q}_{F}^{-1}\mathbf{R}_{\hat{\mathbf{\omega}}_{F}}\left(|\omega_{F}|\Delta{}t\right)
        \right]\left(\mathbf{v}_l+\mathbf{\omega}_{F}\times\mathbf{r}_l\right)
\end{equation}

\subsection{Transformation to Local Frame}
Given the state vector $\langle \mathbf{r}_p, \mathbf{v}_p \rangle$ in the parent frame $P$, and the time since epoch $\Delta{}t$,
elements of the state vector $\langle \mathbf{r}_l, \mathbf{v}_l \rangle$ in the local frame $F$ are
\begin{equation}
    \label{frame_parent_to_local_r}
    \mathbf{r}_l=\left[
            {\mathbf{R}_{\hat{\mathbf{\omega}}_{F}}\left(|\omega_{F}|\Delta{}t\right)}^{\mathrm{T}}\mathbf{Q}_{F}
        \right]\left(\mathbf{r}_{p}-\mathbf{r}_{F}\right)
\end{equation}
\begin{equation}
    \label{frame_parent_to_local_v}
    \mathbf{v}_l=\left[
            {\mathbf{R}_{\hat{\mathbf{\omega}}_{F}}\left(|\omega_{F}|\Delta{}t\right)}^{\mathrm{T}}\mathbf{Q}_{F}
        \right]\left(\mathbf{v}_{p}-\mathbf{v}_{F}\right) - \mathbf{\omega}_{F}\times\mathbf{r}_l
\end{equation}

\subsection{Inertial Frame}
The inertial frame. The orientation is the identity, there is no angular momentum, and the displacement is at the center of the
universal coordinate system.

\begin{equation}
    \langle \left[\mathbf{I}\right], \mathbf{0}, \mathbf{0}, \mathbf{0} \rangle
\end{equation}


\subsection{Perifocal Frame}
The basis of the perifocal frame consist of the vectors $\hat{\mathbf{p}}$, $\hat{\mathbf{q}}$, $\hat{\mathbf{w}}$. These vectors
define the orbital plane. $\hat{\mathbf{w}}$ is in the direction of the normal, $\hat{\mathbf{p}}$ is in the direction of the
periapsis, and $\hat{\mathbf{q}} = \hat{\mathbf{w}} \times \hat{\mathbf{p}}$. $\left[\mathbf{Q}\right]$ can also be derived from
the eular angle of orbital parameters $i$, $\Omega$, and $\omega$.

\begin{equation}
    \label{perifocal_pqw}
    \langle \begin{bmatrix}
        \hat{\mathbf{p}} \\
        \hat{\mathbf{q}} \\
        \hat{\mathbf{w}}
    \end{bmatrix}, \mathbf{0}, \mathbf{0}, \mathbf{0} \rangle
\end{equation}
\begin{equation}
    \label{perifocal_iWw}
    \langle \left[\mathbf{E}\left(\Omega,i,\omega\right)\right]^{\mathrm{T}}, \mathbf{0}, \mathbf{0}, \mathbf{0} \rangle
\end{equation}

\subsubsection{RSW Frame}
\begin{equation}
    \label{perifocal_rsw}
    \langle
    \left[\mathbf{R}_{3}\left(\theta\right)\right]^{\mathrm{T}}
    \begin{bmatrix}
        \hat{\mathbf{p}} \\
        \hat{\mathbf{q}} \\
        \hat{\mathbf{w}}
    \end{bmatrix},
    \mathbf{0},
    \frac{p\left(\cos{\theta}\hat{\mathbf{p}} + \sin{\theta}\hat{\mathbf{q}}\right)}{1+e\cos{\theta}},
    \frac{\mu}{h}\left(-\sin{\theta}\hat{\mathbf{p}} + \left(e+\cos{\theta}\right)\hat{\mathbf{q}}\right)
    \rangle
\end{equation}

\subsubsection{RPW Frame}
\begin{equation}
    \label{perifocal_rpw}
    \langle
    \left[\mathbf{R}_{3}\left(\theta-\gamma\right)\right]^{\mathrm{T}}
    \begin{bmatrix}
        \hat{\mathbf{p}} \\
        \hat{\mathbf{q}} \\
        \hat{\mathbf{w}}
    \end{bmatrix},
    \mathbf{0},
    \frac{p\left(\cos{\theta}\hat{\mathbf{p}} + \sin{\theta}\hat{\mathbf{q}}\right)}{1+e\cos{\theta}},
    \frac{\mu}{h}\left(-\sin{\theta}\hat{\mathbf{p}} + \left(e+\cos{\theta}\right)\hat{\mathbf{q}}\right)
    \rangle
\end{equation}

\subsection{Geocentric Equitorial Frame}
Positions in the geodetic frame are defined by declination ($\delta$), right ascension ($\alpha$), and distance from the center ($r$).
\begin{equation}
    \begin{array}{c c c}
        r = |\mathbf{r}| &
        \delta = \sin^{-1}{\frac{r_z}{r}} &
        \alpha =
        \begin{cases}
            \cos^{-1}{\left(\frac{r_x}{r\cos{\delta}}\right)} & \frac{r_y}{r} > 0 \\
            2\pi-\cos^{-1}{\left(\frac{r_x}{r\cos{\delta}}\right)} & \frac{r_y}{r} \leq 0
        \end{cases}
    \end{array}
\end{equation}
Velocity remains the same, and it's orientation is the same as the inertial frame.

\subsection{Rotating Geocentric Frame}
This frame introduces planitary rotation and inclination. Inclination is defined by $\left[\mathbf{Q}\right]$, and rotation
is defined by the angular velocity $\omega$. The axis of rotation is always $\hat{\mathbf{k}}$.
$\left[\mathbf{Q}\right]$ is the transpose of the eular angle over $\Omega,i,\alpha$.
That is to say $\left[\mathbf{Q}\right]=\left[\mathbf{E}\left(\Omega,i,\alpha\right)\right]^{\mathrm{T}}$ where $\Omega$ is
the right ascension of axial tilt, $i$ is the axial tilt, and $\alpha$ is another angle. %TODO: what is alpha?

\begin{equation}
    \label{rotating_geocentric_frame}
    \langle\left[\mathbf{Q}\right], \omega\hat{\mathbf{k}}, \mathbf{0}, \mathbf{0} \rangle
\end{equation}


\subsection{Topocentric Frame}
The topocentric frame introduces geodetic latitude $\phi$, geocentric latitude $\phi'$, longitude $\beta$, radius at the pole $R_p$,
radius at the equator $R_e$, and altitude $H$.

From this, flattening $f$ and eccentricity $e$ can be derived.
\begin{equation}
    \label{topocentric_flattening}
    f=\frac{R_e-R_p}{R_e}
\end{equation}
\begin{equation}
    \label{topocentric_ecc}
    e=\sqrt{2f-f^2}
\end{equation}


The surface hight $R_\phi$ at any given $\phi$ is
\begin{equation}
    \label{surface_height}
    R_\phi=\frac{R_e}{\sqrt{1-e^{2}\sin^{2}{\phi}}}=\frac{R_e}{\sqrt{1-\left(2f-f^2\right)\sin^2{\phi}}}
\end{equation}


\begin{equation}
    \label{topocentric_r_geocentric}
    \mathbf{r}_f=
    \left(R_{e}+H\right)\left(
        \cos{\phi'}\cos{\beta}\hat{\mathbf{i}} +
        \cos{\phi'}\sin{\beta}\hat{\mathbf{j}} +
        \sin{\phi'}\hat{\mathbf{k}}\right)
\end{equation}
\begin{equation}
    \label{topocentric_r_geodetic}
    \mathbf{r}_f=
    \left(R_\phi+H\right)\cos{\phi}\left(\cos{\beta}\hat{\mathbf{i}} + \sin{\beta}\hat{\mathbf{j}}\right) +
    \left(R_\phi\left(1-f\right)^{2}+H\right)\sin{\phi}\hat{\mathbf{k}}
\end{equation}

\subsubsection{Determining geodetic lattitude ($\phi$)}
\begin{function}
    \DontPrintSemicolon
    \SetKwProg{Def}{def}{:}{end}
    \Def{GodeticLattitudeAltitude($\mathbf{r}$)}{
        $F \leftarrow \lambda\left(\phi,H\right) = \left(R_{\phi}\left(1-f\right)^{2}+H\right)\sin\phi-r_{z}$\;
        $\nabla{}F \leftarrow \lambda\left(\phi,H\right) = \langle H\cos\phi, \sin\phi\rangle$\;
        \Return{$\mathrm{newton}\left(F,
            \langle
                \sin^{-1}\left(\frac{r_{z}}{r}\right),
                r-R_{e}
            \rangle,\nabla{}F\right)$}
    }
    \caption{GeodeticLattitudeAltitude()}
\end{function}

\subsubsection{Horizon Coordinate System}
The horizon coordinate system is defined by the direction cosines $\hat{\mathbf{i}}_h$, $\hat{\mathbf{j}}_h$,
and $\hat{\mathbf{k}}_h$. These vectors form the rows of $\left[\mathbf{Q}_h\right]$.
\begin{equation}
    \label{topocentric_horizon_Q}
    \left[\mathbf{Q}_h\right]=
        \begin{bmatrix}
            -\sin{\beta}           &  \cos{\beta}           & 0          \\
            -\sin{\phi}\cos{\beta} & -\sin{\phi}\sin{\beta} & \cos{\phi} \\
             \cos{\phi}\cos{\beta} &  \cos{\phi}\sin{\beta} & \sin{\phi}
        \end{bmatrix}
\end{equation}

